\section{Model checking algorithms}

\begin{frame}{The CTL model-checking algorithm}
	%Some introduction here...
\end{frame}

\begin{frame}{The CTL model-checking algorithm}
	\framesubtitle{The labelling algorithm}
	We present an algorithm which, given a model and a CTL formula, outputs the set of states of the model that satisfy the formula. 
\end{frame}

\begin{frame}{The CTL model-checking algorithm}
	\framesubtitle{The labelling algorithm}
	The algorithm deals explicitly only with some of the CTL connectives; for the others, it tranforms them to their equivalent form in terms of the minimal set of connectives previously definied: $\{\bot, \neg, \land, \A\F, \E\U, \E\X \}$ 
\end{frame}

\begin{frame}{The CTL model-checking algorithm}
	\framesubtitle{The labelling algorithm}
	Here is the algorithm:\\
	\pause
	{\bf INPUT}: a CTL model $\M = (S, \to, L)$ and a CTL formula $\phi$.\\
	{\bf OUTPUT}: the set of states of $\M$ which satisfies $\phi$.\\
	\pause
	\begin{itemize}
		\item
		{
			First, rewrite $\phi$ in terms of $\bot, \neg, \land, \A\F, \E\U$ and $\E\X$.
			\pause
		}
		\item
		{
			Next, label the states of $\M$ with the subformulas of $\phi$ that are satisfied there, starting with the smallest subformulas and working outwards towards $\phi$.
		}
	\end{itemize}
\end{frame}

\begin{frame}{The CTL model-checking algorithm}
	\framesubtitle{The labelling algorithm}
	Suppose $\psi$ is a subformula of $\phi$ and states satisfying all the immediate subformulas of $\psi$ have already been labelled. We determine by a case analysis which states to label with $\psi$. If $\psi$ is
	
	\begin{itemize}
		\item
		{
			$\bot$: then no states are labelled with $\bot$.
			\pause
		}
		\item
		{
			$p$: then label $s$ with $p$ if $p \in L(s)$.
			\pause
		}
		\item
		{
			$\psi_{1} \land \psi_{2}$: label $s$ with $\psi_{1}$ ? $\psi_{2}$ if $s$ is already labelled both with $\psi_{1}$ and with $\psi_{2}$.
			\pause
		}
		\item
		{
			$\neg \psi_{1}$: label $s$ with $\neg \psi_{1}$ if $s$ is not already labelled with $\psi_{1}$.
			\pause
		}
		\item
		{
			$\A\F \psi_{1}$:
			\begin{itemize}
				\item If any state $s$ is labelled with $\psi_{1}$, label it with $\A\F \psi_{1}$.
				\item Repeat: label any state with $\A\F \psi_{1}$ if all successor states are labelled with $\A\F \psi_{1}$, until there is no change.
			\end{itemize} 
		}
	\end{itemize} 
\end{frame}

\begin{frame}{The CTL model-checking algorithm}
	\framesubtitle{The labelling algorithm}
	Suppose $\psi$ is a subformula of $\phi$ and states satisfying all the immediate subformulas of $\psi$ have already been labelled. We determine by a case analysis which states to label with $\psi$. If $\psi$ is
	
	\begin{itemize}
		\item
		{
			$\E[\psi_{1} \U \psi_{2}]$:
			\begin{itemize}
				\item If any state $s$ is labelled with $\psi_{2}$, label it with $\E[\psi_{1} \U \psi_{2}]$.
				\item Repeat: label any state with $\E[\psi_{1} \U \psi_{2}]$ if it is labelled with $\phi_{1}$ and at least
				one of its successors is labelled with $\E[\psi_{1} \U \psi_{2}]$, until there is no change.
			\end{itemize}
			\pause
		}
		\item
		{
			$\E\X \psi_{1}$: label any state with $\E\X \psi_{1}$ if one of its successors is labelled with $\psi_{1}$.
		}
	\end{itemize}
\end{frame}

\begin{frame}{The CTL model-checking algorithm}
	\framesubtitle{The labelling algorithm}
	\begin{itemize}
		\item
		{
			Having performed the labelling for all the subformulas of $\phi$ (including $\phi$
			itself), we output the states which are labelled $\phi$.
			\pause
		}
		\item
		{
			The complexity of this algorithm is $O(f V (V + E))$, where $f$ is the number of connectives in the formula, $V$ is the number of states and $E$ is the number of transitions.
		}
	\end{itemize}
\end{frame}