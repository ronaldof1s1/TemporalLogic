\section{More about semantics}
\subsection{Equivalences}

\begin{frame}{Equivalences}
	\begin{definition}
		Two CTL formulas $\varphi$ and $\psi$ are said to be \alert{semantically	equivalent} if any state in any model which satisfies one of them also satisfies the other; 
	\end{definition}\pause
	\textbf{Notation:} we denote the equivalence of $\varphi$ and $\psi$ by $\varphi \equiv \psi$
\end{frame}

\begin{frame}{Example of equivalences}
    Let $\varphi$ be an arbitrary CTL formula.
        
    \begin{itemize}
        \item
        {
            $\neg \A\F \varphi \equiv \E\G \neg \varphi $    
            \pause
        }
        \item
        {
            $\neg \E\F \varphi \equiv \A\G \neg \varphi$    
            \pause    
        }
        \item
        {
            $\neg \A\X \varphi \equiv \E\X \neg \varphi$    
            \pause
        }
        \item
        {
            $\A\F\varphi \equiv \A[\top \U\varphi]$    
            \pause
        }
        \item
        {
            $\E\F \varphi \equiv \E[\top \U\varphi]$
        }
    \end{itemize}
\end{frame}

\begin{frame}{Minimum set of CTL connectives}
    Because of the equivalences shown and the ones in propositional logic, we can have some minimum sets of conectives for the CTL syntax. One of them is defined in Backus-Naur formalism below:
    $$\phi ::=  \top \; | \; p \; | \; \neg \phi \; | \; \phi \to \phi \; | \; \A\X \phi \; | \;  \A[\phi\U\phi] \; | \; \E[\phi\U\phi]$$
\end{frame}