
\section{In previous chapters...}
\begin{frame}{Previously on Temporal Logic Week...}
  \framesubtitle{Temporal Logic}
  \begin{itemize}
	\item 
	{
		A brief introduction to Propositional Logic, its syntax and its semantics
		\pause
	}
	\item
	{
		Formal models of time
		\pause
		\begin{itemize}
			\item Frames and Flows of time
			\pause
		\end{itemize}
	}
	\item
	{
		Temporal Logic extends the Propositional Logic
		\begin{itemize}
			\item The connectives $H$ and $G$ \pause
		\end{itemize}
	}
	\item
	{
		Some practical applications
	}
  \end{itemize}

\end{frame}


\begin{frame}{Previously on Temporal Logic Week...}
  \framesubtitle{Model checking}
  \begin{itemize}
	  \item
	  {
		  ``What good is Temporal Logic?"
		  \pause
		  \begin{itemize}
			  \item
			  {
				  Answer: ``Temporal Logic is a good method for specifying and reasoning about a concurrent program".
				  \pause
			  }
		  \end{itemize}
	  }
	  \item
	  {
		  Some mathematical definitions about states of a program and (concurrent) programs
		  \pause
	  }
	  \item
	  {
		  The Peterson's algorithm
		  \pause
	  }
	  \item
	  {
		  How the model checking works?
		  \pause
	  }
	  \item
	  {
		  Some strengths and weaknesses of model checking
	  }
  \end{itemize}
\end{frame}

\begin{frame}{Previously on Temporal Logic Week...}
    \framesubtitle{Linear Temporal Logic}
    \begin{itemize}
        \item Syntax and Semantics of LTL \pause
        
        \item $\omega$-languages, Kripke structures, paths and traces
        
        \item Buchi automata and LTL model checking
    \end{itemize}
\end{frame}