\section{How to communicate}
\subsection{Syntax of CTL}
\begin{frame}{Syntax}
	\framesubtitle{Definition}
	The syntax of CTL consists on the syntax of temporal logic plus some path operators. The class of formulas can be defined in Backus-Naur form. If $\phi$ is a formula: \pause
	$$\phi ::= \bot \; | \; \top \; | \; p \; | \; \neg \phi \; | \; \phi \land \phi \; | \; \phi \lor \phi \; | \; \phi \to \phi \; | \; \phi \leftrightarrow \phi \; | \; \AX \phi \; | \; \EX \phi \; | \; \AF \phi \; |$$
	$$\EF \phi \; | \; \AG \phi \; | \; \EG \phi \; | \; \A[\phi \U \phi] \; | \; \E[\phi \U \phi]$$\pause
	
	With $p$ as a literal (atomic formula), \AX, \EX, \AF, \EF, \AG \, e \EG \, unary operators.
	
\end{frame}

\begin{frame}{Syntax}
	\framesubtitle{Intuition}
	content...
\end{frame}
\subsection{Semanthics of CTL}
\begin{frame}{Semanthics}
	
\end{frame}