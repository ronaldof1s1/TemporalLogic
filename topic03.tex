\section{How to communicate}
\subsection{Syntax of CTL}
\begin{frame}{Syntax}
	\framesubtitle{Definition}
	The syntax of CTL consists on the syntax of temporal logic plus some path operators. The class of formulas can be defined in Backus-Naur form. If $\phi$ is a formula: \pause
	$$\phi ::= \bot \; | \; \top \; | \; p \; | \; \neg \phi \; | \; \phi \land \phi \; | \; \phi \lor \phi \; | \; \phi \to \phi \; | \; \A\X \phi \; | \; \E\X \phi \; | $$
	$$\A\F \phi \; | \; \E\F \phi \; | \; \A\G \phi \; | \; \E\G \phi \; | \; \A[\phi \U \phi] \; | \; \E[\phi \U \phi]$$\pause
	
	With $p$ as a literal (atomic formula), \A\X, \E\X, \A\F, \E\F, \A\G \, e \E\G \, unary operators.
	
\end{frame}

\begin{frame}{Syntax}
	\framesubtitle{Intuition}
	The propositional operators: $\neg, \lor, \land, \to$ have the same meaning of in the propositional logic.\pause
	
	The path-specific operators can be read as: \pause
    \begin{itemize}
		\item 
		{
			\A: is the universal quantifier over paths. Read as: ``in all possible paths'';
			\pause
		}
		\item 
		{
			\E:is the existential quantifier over paths. Read as: ``exists a path in which'';
			\pause
		}
	\end{itemize}
	
    The temporal operators, as in LTL, can be read as:
	\begin{itemize}
    	\item 
    	{
    		$\X$: ``in the next state'';
    		\pause
    	}
    	\item 
    	{
    		$\F$ ``There is some state in the future (eventually)'';
    		\pause
    	}
    	\item
    	{
    		$\G$ ``Globally (in all future states)'';
    		\pause
    	}
    	\item
    	{
    		$\varphi \U \psi$: $\varphi$ is true at least until $\psi$ becomes true;	
    	}
    \end{itemize}

\end{frame}

\begin{frame}{Syntax}
	\framesubtitle{Notes}
	\begin{itemize}
		\item 
		{
			Notice that, in CTL, the combination of path specific operators and temporal operators are atomic, e.g., \A\F \; is an atomic operator that can be read as ``In all paths in the future there is some state where...'';
			\pause
		}
		
		\item 
		{
			Notice as well that the binary operators $\A[\varphi\U\psi]$ and $\E[\varphi\U\psi]$ can be represented as \A\U and \E\U, respectively;
			\pause
		}
		
		\item 
		{
			We assume that, similarly to the $\neg$ operator, the ``new'' unary operators ($\A\X, \E\X, \A\F, \E\F, \A\G $, and $ \E\G $) have the first precedence. Next comes the $\land$ and $\lor$ operators. And at last the $\to$, $\A\U$ and $\E\U$;
		}
	\end{itemize}
	
\end{frame}

\begin{frame}{Examples}
	\begin{itemize}
		\item
		{
			Examples of well-formed formulas:
			\begin{itemize}
				\item $\A\G (p \lor \E\F q)$ \pause
				\item $\A\X (q \to \E[(p\lor q) \U r])$ \pause
				\item $\E\F\E\G p \to \A\F r$ Note that this is binded as $(\E\F\E\G p) \to \A\F r$, not as $\E\F\E\G (p \to \A\F r)$
			\end{itemize}
			\pause
		}
		\item
		{
			Example of formulas that are not well-formed:
			\begin{itemize}
				\item $\A\neg\G \neg p$ \pause
				\item $F[p\U s]$ \pause
				\item $A[p\U s \land q \U s]$
			\end{itemize}
		}
	\end{itemize}
\end{frame}

