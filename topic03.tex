\section{How to communicate}
\subsection{Syntax of CTL}
\begin{frame}{Syntax}
	\framesubtitle{Definition}
	The syntax of CTL consists on the syntax of temporal logic plus some path operators. The class of formulas can be defined in Backus-Naur form. If $\phi$ is a formula: \pause
	$$\phi ::= \bot \; | \; \top \; | \; p \; | \; \neg \phi \; | \; \phi \land \phi \; | \; \phi \lor \phi \; | \; \phi \to \phi \; | \; \A\X \phi \; | \; \E\X \phi \; | $$
	$$\A\F \phi \; | \; \E\F \phi \; | \; \A\G \phi \; | \; \E\G \phi \; | \; \A[\phi \U \phi] \; | \; \E[\phi \U \phi]$$\pause
	
	With $p$ as a literal (atomic formula), \A\X, \E\X, \A\F, \E\F, \A\G \, e \E\G \, unary operators.
	
\end{frame}

\begin{frame}{Syntax}
	\framesubtitle{Intuition}
	The propositional operators: $\neg, \lor, \land, \to$ have the same meaning of in the propositional logic.\pause
	
	The temporal operators can be read (if $\varphi$ is a formula) as follows: \pause
	\begin{itemize}
		\item 
		{
			$\A \phi$: $\varphi$ is true in all possible paths;
			\pause
		}
		\item 
		{
			$\E \phi$: $\varphi$ exists a path in which $\phi$ is true;
			\pause
		}
	\end{itemize}
	
	The path-specific operators can be read, considering $\varphi$ and $\psi$ formulas, as: \pause
	\begin{itemize}
		\item 
		{
			$\X \phi:$ $\varphi$ is  true until next state;
			\pause
		}
		\item 
		{
			$\F \varphi:$ There is some state in the future where $\varphi$ is true;
			\pause
		}
		\item
		{
			$\G \varphi: $ Globally (in all future states) $\varphi$ is true;
			\pause
		}
		\item
		{
			$\varphi \U \psi$: $\varphi$ is true at least until $\psi$ becomes true;	
		}
	\end{itemize}
	
\end{frame}

\begin{frame}{Syntax}
	\framesubtitle{Notes}
	\begin{itemize}
		\item 
		{
			Notice that, in CTL, the combination of path specific operators and temporal operators are atomic, i.e., \A\F \; is a operator that can be read as "In all paths in the future there is some state where..."
		}
		
		\item 
		{
			Notice as well that the binary operators $\A[\varphi\U\psi]$ and $\E[\varphi\U\psi]$ can be represented as \A\U
		}
		
		\item 
		{
			We assume that, similarly to the $\neg$ operator, the "new" unary operators ($\A\X, \E\X, \A\F, \E\F, \A\G $, and $ \E\G $) have the first precedence. Next comes the $\land$ and $\lor$ operators. And at last the $\to$, $\A\U$ and $\E\U$
		}
	\end{itemize}
	
\end{frame}

\subsection{Semanthics of CTL}
\begin{frame}{Semanthics}
	
\end{frame}